\documentclass[a4paper,10pt]{article}
\usepackage[T1]{fontenc}
\usepackage[utf8]{inputenc}
\usepackage{amsmath}
\usepackage{amssymb}
\title{Geometry theory used in program}
\author{Justin Finnerty}
\date{June 2015}
\begin{document}
\maketitle{}

\section{Solid of revolution}

A solid of revolution is obtained when a 2{D} figure
(the \textit{lamina}) is rotated around an axis. Our
simulations axes conform to the standard maths notation
of making such an axis of rotation be along the $z$ axis.

\subsection{Pappus theorems}
\subsubsection{Surface Area}

\begin{equation}
S = sd_i \label{eqn:pappus:surface}
\end{equation}

The surface of area ($S$) of a solid of revolution is equal to
the product of arc length of the generating curve ($s$) the 
distance travelled by the geometric centroid ($d_i$) of 
the lamina.  The geometric centroid is the standard centroid 
for the lamina being rotated.

(Kern and Bland, "Theorem of Pappus" in "Solid Mensuration with
proofs", 2nd ed. New York, Wiley, 1948, 110--115)

\subsubsection{Volume}

\begin{equation}
V = Ad_i \label{eqn:pappus:volume}
\end{equation}

The volume ($V$) of solid of revolution is equal to
the product of area of the lamina ($A$) the 
distance travelled by the geometric centroid ($d_i$) of 
the lamina.  The geometric centroid is the standard 
centroid for the lamina being rotated.

(Kern and Bland, "Theorem of Pappus" in "Solid Mensuration with
proofs", 2nd ed. New York, Wiley, 1948, 110--115)

\subsection{Torus}

The equation of a torus is:

\begin{equation}
(R - \surd{x^2 + y^2})^2 + z^2 = a^2 \label{eqb:torus:cart}
\end{equation}

where $R$ is the distance between the axis of rotation and 
the centre of the lamina circle and $a$ is the radius of
this circle.

In polar coordinates $(u,v \in{} [0,2\pi{}])$:

\begin{align}
x =& (c + a\cos{v})\cos{u} \label{eqn:torus:polar} \\
y =& (c + a\cos{v})\sin{u} \nonumber{}\\
z =& a\sin{v} \nonumber
\end{align}

where $z$ axis is the axis of rotation, the circle 
being rotated is centred on the $xy$ plane and $u$ is the 
angle around the $z$ axis and $v$ is the angle from
the centre of the circle being rotated.

\section{Centroid}

The non-trivial centroids are the ones for walls and arcs.
Even in these cases the centroid in the direction of rotation
($u$ in our polar coordinates) is trivial.

We can calculate a centroid by decomposition into parts:

\begin{equation}
C = \frac{\sum{C_i}{A_i}}{\sum{A_i}} \label{eqn:centroid:area}
\end{equation}

where the centroid ($C$) is the area ($A_i$) weighted average
of the centroids ($C_i$) of the parts:

Dividing the original region into two regions of equal area
(along $v$ coordinate) gives the centroid as the average of
these two sub-centroids. This subdivision can be done ad 
infinitum.

\subsection{Centroid of wall arc from division of area}

Dividing the area of a wall arc starting at $r$ and extending
to radius $r + c$. The centroid radius ($r + a$) will divide the area
into two equal area halves ($A_1, A_0$):

\begin{align*}
A_1 =& \pi(r+c)^2 - \pi(r+a)^2 \\
A_0 =& \pi(r+a)^2 - \pi(r)^2 \\
(r+a)^2 - r^2 =& (r+c)^2 - (r+a)^2 \\
2(r+a)^2 =& (r+c)^2 + r^2 \\
a =&  \frac{\sqrt{(r+c)^2 + r^2}}{\sqrt{2}} - r
\end{align*}

\subsection{Area of arc of rotation}
For an arc centred on the $x$ axis with angle $a$ and
radius $r$ the centroid is given by:

\begin{align}
x =& r\left(\frac{\sin{a/2}}{a/2}\right) \label{eqn:centroid:arc} \\
y =& 0 \nonumber{}
\end{align}

Intuitively the ratio $\frac{\sin{a/2}}{a/2}$ approaches
$1$ as the angle approaches $0$ and $4/\pi{}$ as the angle
approaches $\pi{}/2$.

Converting to our arc we get for arc beginning at $v_0$,
ending at $v_1$ and with $v_1 - v_0 = 2a$:

\begin{align}
x_c =& \left(c + r \frac{\sin{a}}{a} \cos{(v_0 + a)}\right) \cos{u} \\
    =& \left(c + r \frac{\sin{a}}{a} \cos{(v_1 - a)}\right) \cos{u} \\
y_c =& \left(c + r \frac{\sin{a}}{a} \cos{(v_0 + a)}\right) \sin{u} \\
    =& \left(c + r \frac{\sin{a}}{a} \cos{(v_1 - a)}\right) \sin{u} \\
z_c =& r (\sin{(a)}/a) \sin{(v_0 + a)}
\end{align}

Using Pappus theorem (Eqn~\ref{eqn:pappus:surface}) we get

\begin{align}
d_i =& 2\pi{}\left( c + r \frac{\sin{a}}{a} \sin{(v_0 + a)}\right) \\
s =& 4ar \\
A =& 8ar\pi{}\left( c + r \frac{\sin{a}}{a} \sin{(v_0 + a)}\right)
\end{align}

\end{document}
